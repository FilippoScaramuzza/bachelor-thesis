\chapter*{Conclusioni}
% \chapter* -> the introduction isn't the Chapter 1, it's not a numbered chapter
\addcontentsline{toc}{chapter}{Conclusioni} % this line enable the introduction to be listed in the Table Of Contents even if it's not a numbered chapter (see above)
\markboth{}{}

L'obiettivo di questo lavoro di Tesi è stato la realizzazione di un software di simulazione e valutazione di specifici scenari di implementazione del Fog Computing. Lo strumento sviluppato si è dimostrato piuttosto flessibile, permettendo la valutazione dei diversi aspetti architetturali e di configurazione dei sistemi Fog. Il Fog Computing è caratterizzato da un approccio distribuito, derivante dal bisogno di superare i limiti dell'approccio centralizzato del Cloud Computing. Infatti i nodi Fog possono essere posizionati ovunque nella rete tra gli end-node e il Cloud. Questa flessibilità apre molti interrogativi che possono essere fugati tramite software di simulazione come quello realizzato. 

Sono stati ottenuti diversi risultati rilevanti, come l'andamento del successo dell'algoritmo di placement dei servizi al variare della distribuzione della privacy, delle risorse richieste dai servizi, dell'interconnessione tra i nodi a livello Fog, del numero di questi nodi, e del numero di servizi per ogni applicazione. Questo ha permesso di chiarire quali sono i parametri che più influenzano il successo dell'algoritmo, anche valutando le prestazioni durante la simulazione di un sistema così configurato.

La stesura di questa Tesi nasce da un periodo di Internato di Laboratorio nel quale è stato fatto un lavoro di ricerca sugli aspetti trattati nel Capitolo 1 e che sono serviti per la definizione delle simulazioni, della topologia di rete e per l'analisi dei risultati. Il software realizzato, però, si presta a diverse tipologie di analisi: si può pensare ad un'implementazione di un'architettura differente, di un diverso algoritmo di placement, o a diversi tipi di applicazioni che possono essere allocate nella rete. 

L'IoT ha accelerato la cosiddetta ``trasformazione digitale" e fornisce benefici sia ai singoli utenti che alle aziende di diversi settori, come energia, trasporti, istruzione, sanità pubblica e così via. Proprio grazie all'IoT, il numero di dispositivi connessi è in forte crescita e con questo anche la quantità di dati che vengono prodotti.

Il Fog Computing, unitamente alle sue principali estensioni esposte nel Capitolo 1, si offre come una delle soluzioni più promettenti per la gestione dei Big Data che vengono prodotti dai dispositivi IoT.









