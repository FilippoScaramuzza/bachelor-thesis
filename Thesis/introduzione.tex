\chapter*{Introduzione} % \chapter* -> the introduction isn't the Chapter 1, it's not a numbered chapter
\addcontentsline{toc}{chapter}{Introduzione} %this line enable the introduction to be listed in the Table Of Contents even if it's not a numbered chapter (see above)
\markboth{}{}

Nella moderna era dei Big Data il tempo richiesto per accedere ad alcune applicazioni \textit{Cloud-based}, che concentrano l'interna elaborazione dei dati nei \textit{data-center}, potrebbe essere troppo elevato e rendere il paradigma del \textit{Cloud Computing} impraticabile. Inoltre, l'ormai noto incremento dei dispositivi connessi in ambito IoT, ed il relativo rapido aumento dei dati generati nell'edge della rete richiedono l'implementazione del cosiddetto \textit{Cloud-to-Thing Continuum}. In questo ambito sono state avanzate numerose proposte, ad esempio il \textit{Fog Computing}. Con \textit{Fog Computing}, si intendente un'architettura a livello di sistema che distribuisce le funzioni di elaborazione, archiviazione, controllo e rete più vicine agli utenti lungo un \textit{Cloud-to-Thing Continuum}. I possibili ambiti di applicazione del \textit{Fog Computing} sono innumerevoli: dagli \textit{Smart Vehicles} e il \textit{Traffic Control}, alle \textit{Smart Cities} e agli \textit{Smart Buildings}.

Questo lavoro di Tesi nasce da un periodo di Internato di Laboratorio presso l'Università di Parma, il cui obiettivo è stato quello di realizzare un sistema di simulazione di sistemi Fog, con lo scopo di poter analizzare le prestazioni di questo paradigma in un'architettura \textit{N-Tier}, studiando anche l'implementazione di uno specifico algoritmo di placement dei servizi.

L'architettura \textit{N-Tier di riferimento}, ecc....