\chapter*{Introduzione} % \chapter* -> the introduction isn't the Chapter 1, it's not a numbered chapter
\addcontentsline{toc}{chapter}{Introduzione} %this line enable the introduction to be listed in the Table Of Contents even if it's not a numbered chapter (see above)
\markboth{}{}

Nell'era dei Big Data il tempo richiesto per accedere ad alcune applicazioni \textit{Cloud-based}, che concentrano l'intera elaborazione dei dati nei \textit{data-center}, potrebbe essere troppo elevato. Inoltre, l'ormai noto incremento dei dispositivi connessi in ambito Internet of Things, ed il conseguente rapido aumento dei dati generati nell'edge della rete richiedono l'implementazione del cosiddetto \textit{Cloud-to-Thing Continuum}. In questo ambito sono state avanzate numerose proposte, ad esempio il \textit{Fog Computing}, un paradigma a livello di sistema che distribuisce le funzioni di elaborazione, archiviazione, controllo e comunicazione dall'edge al Cloud. \cite{OpenFogReferenceArchitecture}. I possibili ambiti di applicazione del \textit{Fog Computing} sono innumerevoli: dagli \textit{Smart Vehicles} e il \textit{Traffic Control}, alle \textit{Smart Cities} e agli \textit{Smart Buildings}.

Questo lavoro di Tesi nasce da un periodo di Internato di Laboratorio presso l'Università di Parma, il cui obiettivo è stato quello di realizzare simulazioni di sistemi Fog, con lo scopo di poter analizzare le prestazioni di questo paradigma in un'architettura \textit{N-Tier}, studiando anche l'implementazione di uno specifico algoritmo di placement dei servizi.

L'architettura di riferimento ha una struttura composta da 3 macro-entità: il Cloud, il livello Fog e gli end-node/dispositivi IoT. Il livello Fog, inoltre, è ulteriormente scomposto in sotto-livelli (tier) che più sono distanti dai dispositivi, più aumentano le loro capacità computazionali.

Il software realizzato si basa sul simulatore \textit{YAFS} (\textit{Yet Another Fog Simulator}), un simulatore ad eventi discreti sviluppato in Python ed a sua volta basato su \textit{SimPy}, ovvero un framework DES (\textit{Discrete Event Simulator}) basato sui processi, anch'esso sviluppato in Python.

Nel \textit{Capitolo 1} vengono introdotti i principali concetti fondanti del \textit{Cloud Computing}, del \textit{Fog Computing} e degli altri principali paradigmi in quest'ambito di ricerca. Vengono poi citate possibili applicazioni del \textit{Fog Computing} e viene fatta una breve introduzione a YAFS.

Nel \textit{Capitolo 2} vengono affrontati i dettagli implementativi del simulatore YAFS, la sua struttura e la modellazione delle simulazioni. Successivamente vengono approfondite le caratteristiche degli scenari simulati in questo lavoro di Tesi.

Nel \textit{Capitolo 3} vengono trattati i principali aspetti del sistema di simulazione realizzato. Vengono poi discusse le modalità di utilizzo e le analisi che si possono eseguire.

Nel \textit{Capitolo 4}, vengono presentati i principali risultati ottenuti dalle simulazioni eseguite. Vengono in particolare valutati l'algoritmo di placement utilizzato al variare di specifici parametri di configurazione del sistema ed il soddisfacimento delle richieste con e senza failure control.

In conclusione, vengono discussi il lavoro svolto e i possibili sviluppi futuri.


